% /* cSpell:enable */
% /* cSpell:locale de,en */
\chapter{Schlussbemerkungen}
\label{cha:Schluss}

Im Rahmen dieser Arbeit wurde ein Python-Skript mit Jupyter Code Cells zur Abfrage und Ausgabe der Daten sowie Beschreibungen der Datensätze entwickelt und erstellt. Diese Arbeit wurde zur freien Verwendung auf GitHub veröffentlicht. Auch wenn es bereits zahlreiche Dashboards gibt, so sollte es auch Open-Source-Anwendungen geben, die Anwendern ermöglichen, den Quellcode einzusehen und sich selber mit Datensätzen aus verschiedenen Quellen auseinanderzusetzen.
Dies soll auch weiterführende Arbeiten ermöglichen.

Es wurden Korrelationen sowohl zwischen Maßnahmen als auch zwischen Impfquote und Fall- sowie Todeszahlen gefunden. Herauszufinden, welche Maßnahmen genau Wirkung zeigten, stellte sich als schwierig heraus, da meist mehrere Maßnahmen gleichzeitig verordnet wurden die Abstände zwischen neuen Verordnungen oft zu kurz waren, um eine Veränderung mit Sicherheit zu belegen.

Die Frage nach der Wirksamkeit wird auch weiterhin eine Komponente der individuellen Einschätzung beinhalten. Unter den Befragten spielte dabei die Bildung keine Rolle, wie die Wirksamkeit eingeschätzt wird. Geimpfte Personen halten Maßnahmen und die Impfung für wirksamer und haben mehr Vertrauen in die Impfung als nicht geimpfte Personen.

Die Umfrage sowie die quantitativen Rohdaten wurden ebenfalls zur Einsicht veröffentlicht. Mit dieser Arbeit soll nun auch die Auswertung derselben für die Öffentlichkeit frei einsehbar sein.
% /* cSpell:disable */