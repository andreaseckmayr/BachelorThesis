% /* cSpell:enable */
% /* cSpell:locale de,en */
\chapter{Kurzfassung}

In den letzten zweieinhalb Jahren haben wir uns alle zwangsweise mit Statistiken und dem Begriff Wirksamkeit auseinandersetzen müssen. Es existieren viele unterschiedliche Auffassungen darüber, was Wirksamkeit bedeutet. Diese Arbeit hat sich zum Ziel gesetzt, den Begriff der Wirksamkeit zu erläutern und zu veranschaulichen, aber auch die Frage, welche Maßnahmen und Impfstrategieen wirksam und erfolgreich waren, anhand zur Verfügung stehender Daten zu klären.

Im praktischen Teil der Arbeit wurden Zahlen aus verschiedenen Datensätzen verglichen und untersucht, ob sich daraus eine Wirksamkeit des Impfstoffes bzw. eine Wirksamkeit der Maßnahmenregelungen empirisch belegen lässt. Es wurden Korrelationen sowohl zwischen den verordneten Maßnahmen als auch zwischen der Impfquote und den Erkrankungs- bzw. Todesfällen gefunden.

Im Zuge der Arbeit wurde eine Onlinebefragung unter 425 TeilnehmerInnen durchgeführt, die dazu diente, die aktuellen Ansichten und Beweggründen der Bevölkerung abzubilden und mit den Zahlen und Fakten zu vergleichen. Es zeigte sich, dass die Beantwortung der Frage der Wirksamkeit vom Impfstatus und der Einstellung Impfungen gegenüber, nicht aber von der Bildung abhängig ist.
% /* cSpell:disable */